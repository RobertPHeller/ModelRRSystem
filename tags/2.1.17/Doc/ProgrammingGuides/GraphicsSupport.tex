%* 
%* ------------------------------------------------------------------
%* GraphicsSupport.tex - Using the graphics support code.
%* Created by Robert Heller on Thu Apr 19 14:38:52 2007
%* ------------------------------------------------------------------
%* Modification History: $Log$
%* Modification History: Revision 1.2  2007/11/30 13:56:50  heller
%* Modification History: Novemeber 30, 2007 lockdown.
%* Modification History:
%* Modification History: Revision 1.1  2007/05/06 12:49:38  heller
%* Modification History: Lock down  for 2.1.8 release candidate 1
%* Modification History:
%* Modification History: Revision 1.1  2002/07/28 14:03:50  heller
%* Modification History: Add it copyright notice headers
%* Modification History:
%* ------------------------------------------------------------------
%* Contents:
%* ------------------------------------------------------------------
%*  
%*     Model RR System, Version 2
%*     Copyright (C) 1994,1995,2002-2005  Robert Heller D/B/A Deepwoods Software
%* 			51 Locke Hill Road
%* 			Wendell, MA 01379-9728
%* 
%*     This program is free software; you can redistribute it and/or modify
%*     it under the terms of the GNU General Public License as published by
%*     the Free Software Foundation; either version 2 of the License, or
%*     (at your option) any later version.
%* 
%*     This program is distributed in the hope that it will be useful,
%*     but WITHOUT ANY WARRANTY; without even the implied warranty of
%*     MERCHANTABILITY or FITNESS FOR A PARTICULAR PURPOSE.  See the
%*     GNU General Public License for more details.
%* 
%*     You should have received a copy of the GNU General Public License
%*     along with this program; if not, write to the Free Software
%*     Foundation, Inc., 675 Mass Ave, Cambridge, MA 02139, USA.
%* 
%*  
%* 

\chapter{Using the Graphics Support Code.}
\label{chapt:GraphicsSupport}
\typeout{$Id}

Version 2 of the Graphics Support is covered in this chapter.  Version
1 is depreciated and is included only for older applications.

The grsupport package defines the constants $\pi$ and $\pi \over 2$, a
procedure to convert degrees to radians, plus a collection of Snit
macros that define common validating methods that are useful when
defining Snit types.

\lstinputlisting[caption={Graphics Support package examples},
		   label={lst:GRS:Examples},
		   firstline=38]{SampleCodeCircle.tcl}
\begin{figure}[hbpt]
\begin{centering}
\includegraphics{Circles.png}
\caption{Circles drawn with the Circle type.}
\label{fig:GRS:Circles}
\end{centering}
\end{figure}
Listing~\ref{lst:GRS:Examples} shows some code that uses the Graphics
Support package.  This code defines a Snit type named Circle, which extends
the oval item type for canvas drawing.  The type draws an oval, and
then binds several events to the oval. Figure~\ref{fig:GRS:Circles}
shows a pair of circles drawn with the code in
Listing~\ref{lst:GRS:Examples}.

This code uses several constants and macros defined in the
Graphics Support package, including code to verify option values and the
constant $\pi$.


